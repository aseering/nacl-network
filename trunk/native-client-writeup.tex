\documentclass[a4paper,10pt]{article}
\usepackage[utf8x]{inputenc}
\usepackage{url}

%opening
\title{POSIX Network Support for Google Native Client}
\author{Nizam Ordulu, Adam Seering}

\begin{document}

\maketitle

\begin{abstract}
This project provides secure support for POSIX networking API calls within the
untrusted Google Native Client environment.

The specific implementation prevents users of Google Native Client from being
used for reflection attacks against a remote server, and protects users'
private firewalled network services from being exposed to an attacker on the
other side of the firewall.
\end{abstract}

\section{Native Client System Calls}


Native Client (NaCl) is an open source browser plugin project spearheaded by
Google. Its aim is to give the web applications the performance of native
applications without comprimising safety. It provides a sandbox that runs x86
code with certain restrictions. These restrictions guarantee that the x86
modules can be disassembled almost entirely, the module's control flow is
predictable, and only the allowed system calls are performed. Our work is
closely related to system calls, and in order to understand why our
modifications do not break the security of NaCl, it is important to understand
how NaCl prevents direct access to system calls.

In order to prevent the untrusted x86 module from accessing system calls
directly, NaCl makes use of x86 memory segments and page protections. Upon
loading, the NaCl module is allocated a 256 MB memory region. The first 64 KBs
of this region is initialized by NaCl's service runtime. The first 4KBs of this
64KBs are reserved space to detect NULL pointers. The remaining 60KB contains
the code for the trusted trampoline call gate and springboard return gate.
Following this 64KB lies the text of the x86 module's binary. Segment registers are set to
constrain data accesses from this region to memory with higher addresses. The
trampoline and springboard code region is allowed to access other trusted code
because they are placed by NaCl's service runtime. NaCl applications make
system calls through this trampoline and springboard code region.

There are two types of modifications to NaCl source code that we did: adding
new system calls that are wrappers to POSIX socket system calls and adding
standard socket utility functions like  \texttt{htons, inet\_pton, inet\_addr} 
etc that only do computation to untrusted code's standard library. Assuming
that NaCl's trampoline/springboard mechanism works correctly, this means that
attackers can not access the socket system calls directly. We discuss our
threat model and how we prevent attackers from abusing the network in the next
section.

\section{Threat Model}

Our restrictions to NaCl application's access to network are stricter than
browser's restrictions to javascript applications. By means of an out-of-band
handshake, we require that the other party expresses its willingness to
communicate with the NaCl application before establishing any connection. This
prevents any DDOS attacks or unauthorized accesses to internal network
services. This handshake happens outside the sandbox, therefore can not be
bypassed by the untrusted code.

\section{API}

The POSIX Networking API is replicated using Native Client Syscalls.  The
Native Client Syscalls call functions in Native Client's trusted code library,
which can call the equivalent operating-system-level syscalls after an
appropriate amount of checking.

A number of POSIX API calls, as follows, are passed to Native Client this way.
The following function calls are passed through unmodified:

\begin{itemize}
\item \texttt{getaddrinfo(const char *node,
                const char *service,
                const struct addrinfo *hints,
                struct addrinfo **res)}
\item \texttt{int getsockopt(int socket, int level, int option\_name, const void
                *option\_value, socklen\_t option\_len)}
\item \texttt{int listen(int socket, int backlog)}
\item \texttt{ssize\_t recv(int socket, void *buffer, size\_t length, int
flags)}
\item \texttt{ssize\_t recvmsg(int socket, struct msghdr *message, int flags)}
\item \texttt{ssize\_t send(int socket, const void *buffer, size\_t length, int
flags)}
\item \texttt{ssize\_t sendmsg(int socket, const struct msghdr *message, int
flags)}
\item \texttt{int setsockopt(int socket, int level, int option\_name, const void
                *option\_value, socklen\_t option\_len)}
\item \texttt{int shutdown(int socket, int how)}
\item \texttt{int socket(int domain, int type, int protocol)}
\item \texttt{int socketpair(int domain, int type, int protocol,
       int socket\_vector[2])}
\end{itemize}

None of these function calls are able to call any harm by themselves (aside
from resource exhaustion on the client, which Native Client applications can
already achieve in other ways).  They simply gather information from the system
or allocate local resources for use by other system calls.

The following functions did require modification:

\begin{itemize}
\item \texttt{int accept(int s, struct sockaddr *addr, socklen\_t *addrlen)}
\item \texttt{int connect(int socket, const struct sockaddr *address,
       socklen\_t address\_len)}
\item \texttt{int bind(int socket, const struct sockaddr *address,
       socklen\_t address\_len)}

\item \texttt{int getpeername(int socket, struct sockaddr *restrict address,
       socklen\_t *restrict address\_len)}
\item \texttt{int getsockname(int socket, struct sockaddr *restrict address,
       socklen\_t *restrict address\_len)}

\item \texttt{ssize\_t recvfrom(int socket, void *restrict buffer, size\_t
length,
       int flags, struct sockaddr *restrict address,
       socklen\_t *restrict address\_len)}
\item \texttt{ssize\_t sendto(int socket, const void *message, size\_t length,
       int flags, const struct sockaddr *dest\_addr,
       socklen\_t dest\_len)}
\end{itemize}

In particular, each of these functions has the ability to configure the IP
address that a socket points at, either directly or indirectly by using
the socket to perform some action or discover some information.  This
information is contained within the ``struct sockaddr'' structure.  The
modification to each of these calls was very straightforward: Wrapper code added
to Native Client's trusted library, first verifies that the host at the
IP address specified in the ``struct sockaddr'' argument is willing to
communicate with this client.  If the host is willing to communicate, control
is transparently passed to the underlying operating-system-level protocol.  If
the host is not willing to communicate, the Native Client system call returns
with an error code of $-1$, a generic error code, so that sandboxed
applications can't trivially tell whether a host refused the connection, or was
simply unavailable.

\section{Protocol}

The verification protocol takes place using an out-of-band TCP socket
connection between the client and the remote host.  The verification is
executed entirely in trusted Native Client code, and takes place synchronously
(so that it appears atomic from the perspective of the Native Client code) so
that the untrusted sandbox can't insert malicious responses on behalf of the
remote server.

An out-of-band connection method was chosen so that both client and server
applications can be run without modification, aside from any modifications that
the client otherwise needs to run within Google Native Client.  It allows
further communications to occur over unmodified IP.

\subsection{Validation}

In order for a connection to be allowed to take place, it must first match a
number of conditions.

We first verify that the connection is taking place over IP.  We do not allow
non-IP connections at this time, and we restrict IP connections to IPv4.
Additional protocols could be allowed securely without a great deal of
difficulty; future development could focus on adding such support.

Once we have confirmed that the connection is taking place over IP, if we have
not seen the requested IP addres before, we open a TCP connection to port 1123
on the IP address that the Native Client application is attempting to connect
to, and we send a packet containing the SHA-1 hash of the binary .nexe file
containing the Native Client application, and a randomly-generated nonce.  If
this connection is refused, the Native Client application's attempted connection
is aborted.

If the connection is accepted, the server returns a status integer, the
original nonce generated by the client, and a bitmap of IP ports that the
client is allowed to connect to.  If the status integer is nonzero, the
client's connection attempt is aborted; this allows the server to reject
clients with invalid hashes.  If the returned nonce is not equal to
the original nonce sent by the client, the connection is aborted; this
prevents timed replay attacks by an attacker seeking to cause an application
to connect to a server on a disallowed port.  If the bit in
the bitmap corresponding to the port that the Native Client application wants
to connect to, is set to 1, the connection is allowed to proceed.

We keep an in-process cache of (IP, Allowed Ports) mappings, so that this
information does not need to be retrieved multiple times from the remote
server.  We do not currently provide a means for the remote server to
invalidate this cache; server administrators are expected to provide correct
initial port-access information, and to instruct their users to close and
re-open Google Native Client if they need to change that information rapidly.
We choose not to provide a cache-invalidation mechanism because such a
mechanism would put additional load on the remote server; it would either need
to respond to many more validation requests, or maintain open sockets to each
connected client.

\section{Usage}

Our code is available on Google Code, project name ``nacl-network'':

\url{http://code.google.com/p/nacl-network}

It can be downloaded from the ``Source'' tab at that location.

As our code is based heavily on Google Native Client, it uses the Google Native
Client's build system, as documented on the Native Client page:

\url{
http://nativeclient.googlecode.com/svn/trunk/src/native_client/documentation/bui
lding.html}

We use the following (modified) sequence of build commands within their build
system:

\begin{verbatim}
cd nacl-network/native_client/
./scons
./scons --mode=nacl_extra_sdk,dbg-linux extra_sdk_clean
extra_sdk_update_headerextra_sdk_update
./scons --mode=nacl
yes | ./scons firefox_install
firefox
"http://localhost/native_client/scons-out/nacl-x86-32/staging/echo.html"
\end{verbatim}

Alternatively, if you are running Linux (particularly Ubuntu Jaunty with
Firefox 3.0, though other versions may work), you can try to install by
un-tarring ``native-client-plugin-linux.tar.gz'' from the root of the code
repository, into ``~/.mozilla/plugins/''.  (The tarball outputs all of the
plugin files directly, it doesn't create a containing folder.)  Note that the
plugin may require you to install libSDL.

Note that the test above requires that you are serving the Native Client
repository over HTTP from your local computer.  This requirement is imposed
by Native Client's security model, which is currently very conservative while
Native Client is under development. Native Client comes bundled with a
Python-based web server that you can use for this purpose, if you are not
already running a Web server; its usage is documented at the link above.

Also note that this series of build commands requires the Firefox Web browser,
and will install our modified Native Client plugin for Firefox.

Our code modifications have been tested on Linux; specifically, 32-bit Ubuntu
Jaunty.  (Native Client makes use of the memory segmentation features of
the 32-bit x86 processor, and so is unable to run natively under 64-bit Linux.)
Google Native Client also supports Mac OS X and Windows. We would expect our
code to support Mac OS X with relatively little porting effort. We would expect
a Windows port to take somewhat more effort, as Windows has a substantially
different underlying networking subsystem.

The test case above also requires that an echo server be running on
localhost:7 (such a server is available through xinetd under Linux), and for a
validation server to be running on port 1123 to authorize clients to connect.

\subsection{Validation Server}

We provide a simple validation server as an example for server maintainers
attempting to make use of this patch.  The server can be found at
\texttt{third\_party/nacl\_acl\_server/} inside the source repository.  The
server can be built using the Makefile in that directory.  See the README file
in that directory for further details.

The provided validation server reads, but does not perform any checks on, the
hash of the client application.  This is expected to be the common case for
public services accessed by a wide array of applications.  Private services
will want to insert a check in this code to verify the hash, and set the ``ok''
variable to a nonzero value if the hash is not recognized.  The server also
allows ports 7 (echo), 80 (http), and 1123 (Native Client validation).  In
general, servers will probably not want to allow client programs to access port
1123 on production servers, so as to restrict their access to authentication
information. However, for debugging and development purposes this setting can be
quite useful.

\section{Credits}

In addition to making extensive use of the Google Native Client application,
source code, and documentation, this project also uses header files from
the glibc library,  as well as source files that provide some low-level network
byte-ordering and IP-address-parsing functionality.  It also makes use of the
sample ``echo'' client and server code from the book \underline{TCP/IP Sockets
in C}, by Kenneth L. Calvert and Michael J. Donahoo.

\end{document}
